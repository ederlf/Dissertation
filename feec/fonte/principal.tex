%% abtex2-modelo-trabalho-academico.tex, v-1.7.1 laurocesar
%% Copyright 2012-2013 by abnTeX2 group at http://abntex2.googlecode.com/
%%
%% This work may be distributed and/or modified under the
%% conditions of the LaTeX Project Public License, either version 1.3
%% of this license or (at your option) any later version.
%% The latest version of this license is in
%%   http://www.latex-project.org/lppl.txt
%% and version 1.3 or later is part of all distributions of LaTeX
%% version 2005/12/01 or later.
%%
%% This work has the LPPL maintenance status `maintained'.
%%
%% The Current Maintainer of this work is the abnTeX2 team, led
%% by Lauro C\'{e}sar Araujo. Further information are available on
%% http://abntex2.googlecode.com/
%%
%% This work consists of the files abntex2-modelo-trabalho-academico.tex,
%% abntex2-modelo-include-comandos and abntex2-modelo-references.bib
%%

% ------------------------------------------------------------------------
% ------------------------------------------------------------------------
% abnTeX2: Modelo de Trabalho Academico (tese de doutorado, dissertacao de
% mestrado e trabalhos monograficos em geral) em conformidade com
% ABNT NBR 14724:2011: Informacao e documentacao - Trabalhos academicos -
% Apresentacao
% ------------------------------------------------------------------------
% ------------------------------------------------------------------------

\input{preambulo}

% ---- compila o \'{\i}ndice  ----
\makeindex
\makenomenclature
% ---

% ---- In\'{\i}cio do documento ----
\begin{document}

% Retira espa\c{c}o extra obsoleto entre as frases.
\frenchspacing

% ---- ELEMENTOS PR\'{E}-TEXTUAIS ----
\pretextual

\pagenumbering{roman}

% --- Capa ---
\imprimircapa
% ---

% --- Folha de rosto (o * indica que haver\'{a} a ficha catalogr\'{a}fica) ---
\setcounter{page}{3}
\imprimirfolhaderosto*
% ---

% --- Inserir a ficha catalogr\'{a}fica ---

% Isto \'{e} um exemplo de Ficha Catalogr\'{a}fica, ou ``Dados internacionais de
% cataloga\c{c}\~{a}o-na-publica\c{c}\~{a}o''. Voc\^{e} pode utilizar este modelo como refer\^{e}ncia.
% Por\'{e}m, provavelmente a biblioteca da sua universidade lhe fornecer\'{a} um PDF
% com a ficha catalogr\'{a}fica definitiva ap\'{o}s a defesa do trabalho. Quando estiver
% com o documento, salve-o como PDF no diret\'{o}rio do seu projeto e substitua todo
% o conte\'{u}do de implementa\c{c}\~{a}o deste arquivo pelo comando abaixo:

% --- Para a vers\~{a}o final, delete as linhas abaixo e insira a linha do \includepdf
 \begin{fichacatalografica}
    \vspace*{\fill}
    \begin{center}
        \textsc{Inclua aqui o pdf com a ficha catalogr\'{a}fica fornecida pela BAE.}
    \end{center}
    \vspace*{\fill}
% --- --- ---
    %\includepdf{ficha-catalografica.pdf}
 \end{fichacatalografica}
% ---


% --- Inserir folha de aprova\c{c}\~{a}o ---

% Isto \'{e} um exemplo de Folha de aprova\c{c}\~{a}o, elemento obrigat\'{o}rio da NBR
% 14724/2011 (se\c{c}\~{a}o 4.2.1.3). Voc\^{e} pode utilizar este modelo at\'{e} a aprova\c{c}\~{a}o
% do trabalho. Ap\'{o}s isso, substitua todo o conte\'{u}do deste arquivo por uma
% imagem da p\'{a}gina assinada pela banca com o comando abaixo:

% --- Na vers\~{a}o final, exclua essas linhas e insira o \includepdf
\newpage
\vspace*{\fill}
\begin{center}
    \textsc{Inclua aqui a folha de assinaturas.}
\end{center}
\vspace*{\fill}
\newpage
% --- --- ---
%\includepdf[pagecommand={\thispagestyle{plain}}]{folha-assinaturas.pdf}	
\cleardoublepage


% ---

\begin{resumo}
    OpenFlow is the most prominent technology to enable Software Defined Networking. Designed as control interface between switches and controllers, the protocol can be considered an instruction set to program the network forwarding logic. The first OpenFlow version attracted attention from both the industry and academy researchers interested in SDN promised benefits. Quickly, a toolset for OpenFlow 1.0 was available, which included switches, controllers, test and emulation software. When the protocol standardization process started by the Open Network Foundation, OpenFlow evolved fast and new specifications emerged in the last years. New features empowered the protocol and created enthusiasm, however projects of experimentation tools did not followed the OpenFlow fast pace. This work addresses one of these gaps, implementing an experimenter friendly OpenFlow 1.3 software switch. Driven by simplicity and basic performance requirements, the tool purpose is to be a functional and easy option for SDN developers that want to take advantage of the benefits brought by more recent OpenFlow versions. Overall, this project resulted in the open source release of the first OpenFlow 1.3 switch, allowing researchers from all around the globe to prototype and demonstrate solutions impossible in a prior moment. 
    \vspace{\onelineskip}


    \noindent\textbf{Keywords}: Computer Networks; Software Defined Networking; OpenFlow; Future Internet.

    % \vspace{\onepageskip}
    % \vspace{\onelineskip}
    \pagebreak

    \begin{otherlanguage*}{brazilian}
    \begin{center}{\ABNTEXchapterfont\huge Resumo}\end{center}
        \textit{OpenFlow} é a mais proeminente tecnologia para a implementaç\~{a}o de Redes Definidas por Software (RDS). Projetada como uma interface de controle entre switches e controladores, o protocolo pode ser visto como um conjunto de instruç\~{o}es para programar a l\'{o}gica de encaminhamento em comutadores da rede. A primeira vers\~{a}o do \textit{OpenFlow} atraiu a atenção de pesquisadores da ind\'{u}stria e universidades interessados nos potenciais benef\'{i}cios prometidos por RDS. R\'{a}pidamente surgiram ferramentas para experimentação em \textit{OpenFlow} 1.0, incluindo comutadores, controladores e software para testes e emulaç\~{a}o. Ap\'{o}s o in\'{i}cio da padronizaç\~{a}o do protocolo pela \textit{OpenNetworkFoundation}, o protocolo OpenFlow evoluiu rapidamente dando origem \`{a} novas especificaç\~{o}es. As novas funcionalidades aumentaram as possibilidades de experimentos, gerando entusiasmo. Porém, o desenvolvimento das ferramentas de experimentaç\~{a}o n\~{a}o acompanharam o mesmo r\'{i}tmo do protocolo. Para preencher essa lacuna, esse projeto desenvolveu um comutador em \textit{software} com suporte a \textit{OpenFlow} 1.3. Guiado pelo objetivo de ser simples e básicos requerimentos de performance, a proposta da ferramenta \'{e} ser uma opç\~{a}o, f\'{a}cil e funcional para desenvolvedores de aplicaç\~{o}es RDS  buscando utilizar as novas funcionalidades do \textit{OpenFlow} 1.3. Em suma, o \textit{software} desenvolvido nesse projeto foi o primeiro comutador \textit{OpenFlow} 1.3 do mundo. Lançado como projeto de c\'{o}digo aberto, possibilitou a pesquisadores de todo o mundo a prototipagem e demonstraç\~{a}o de soluç\~{o}es n\~{a}o poss\'{i}veis anteriormente.

    \vspace{\onelineskip}

    \noindent\textbf{Palavras-chaves}: Redes de Computadores, Software Defined Networking, OpenFlow, Internet do Futuro. 

    \end{otherlanguage*}
\end{resumo}

% --- RESUMOS (em ingl\^{e}s ---
% \begin{resumo}

%     \begin{otherlanguage*}{english}   
%     \lipsum[2]

%     \vspace{\onelineskip}

%     \noindent\textbf{Keywords}: Computer Networks; Software Defined Network; OpenFlow; Internet of Future.

%     \end{otherlanguage*}
% \end{resumo}
% ---

% --- inserir o sumario ---
\pdfbookmark[0]{\contentsname}{toc}
\tableofcontents*
\cleardoublepage
% ---

% --- Dedicat\'{o}ria ---
\begin{dedicatoria}
    \vspace*{\fill}
    \centering
    \noindent
    \textit{The road for a dream is not a solitary path. During this arduous and long walk, some faces keep up by your side until the end, while others come and go in the middle of the way. Regardless the time spent with you, everyone leaves marks and contribute, for the good and for the bad, with your personal growth. For this reason, I would like to dedicate this work to everyone that at some point in my life, helped me to go through this process and reach my aspirations.} 
    
    \vspace*{\fill}
\end{dedicatoria}
% ---

% --- Agradecimentos ---
\begin{agradecimentos}
   
    My thanks to my parents, for all the support, affection and belief. Also for raising me with enough freedom to chose my own path. I am grateful for my family too, for the pride they've always kept for my achievements, always motivating me to go further. 

    Christian Esteve Rothenberg, my supervisor and friend. 

    My friends, C\'{a}ssio, Daniel and M\^{o}nica, for the company, laughs and nonsense conversations that made things lighter. 
    
    Allan Vidal, for the long work partnership during all this years and due to the implementation of multiple connections in the software switch and all the help with~\LaTeX~figures.

    I am glad to have meet Kathrin. She showed me the light when everything was getting dark.

    Marcos Rog\'{e}rio Salvador and Marcelo Ribeiro Nascimento, people who guided me through the SDN world and made me believe in the possibility to achieve international recognition.
    
    Many thanks for CPqD, the Brazilian research center where the project development started; Ericsson Innovation Center Brazil, for sponsoring this work, and Ericsson Traffic Lab in Hungary, for technical support. 

    Among the collaborators I would like to thank and highlight two: Zoltán Lajos Kis, for the OpenFlow 1.1 software switch implementation and technical guidance; Jean Tourrilhes, for critical bug fixes and help with the git workflow.

    Finally, my sincerely acknowledgments to people who collaborated with code, bug reports or suggestions.
    
    
\end{agradecimentos}
% ---

% --- Ep\'{\i}grafe  ---
\begin{epigrafe}
    \vspace*{\fill}
	\begin{flushright}
		\textit{``The path to OpenFlow is not a four lane highway of joy and freedom with a six pack and a girl in the seat next to you, it’s a bit more complex and a little hard to say how it will work out, but I’d be backing OpenFlow in my view''\\
		Greg Ferro}
	\end{flushright}
\end{epigrafe}
% ---


% --- inserir lista de ilustra\c{c}\~{o}es ---
\pdfbookmark[0]{\listfigurename}{lof}
\listoffigures*
\cleardoublepage
% ---

% --- inserir lista de tabelas ---
\pdfbookmark[0]{\listtablename}{lot}
\listoftables*
\cleardoublepage
% ---

% --- inserir lista de Acronimos e Abrevia\c{c}\~{o}es ---
\renewcommand{\nomname}{Acronyms and Abbreviations List}
\pdfbookmark[0]{\nomname}{las}
\nomenclature{ARP}{Address Resolution Protocol}
\nomenclature{API}{Application Programming Interface}
\nomenclature{ASIC}{Application Specific Integrated Circuit}
\nomenclature{BOS}{Bottom of the stack}
\nomenclature{CAM}{Content Addressable Memory}
\nomenclature{CRC}{Cyclic Redundancy Check}
\nomenclature{DSCP}{Differentiated Services Code Point}
\nomenclature{EH}{Extension Header}
\nomenclature{Kbps}{Kilobits per second}
\nomenclature{Kpps}{Kilo packets per second}
\nomenclature{ICMP}{Internet Control Message Protocol }
\nomenclature{IP}{Internet Protocol}
\nomenclature{IPv6}{Internet Protocol Version 6}
\nomenclature{JSON}{JavaScript Object Notation}
\nomenclature{LXC}{Linux Containers}
\nomenclature{MAC}{Media Access Control}
\nomenclature{Mbps}{Megabits per second}
\nomenclature{MIPS}{Microprocessor without Interlocked Pipeline Stages)}
\nomenclature{MPLS}{Multiprotocol Label Switching}
\nomenclature{NetPDL}{Network Protocol Description Language}
\nomenclature{ONF}{Open Network Foundation}
\nomenclature{ONOS}{Open Network Operating System}
\nomenclature{OVS}{Open vSwitch}
\nomenclature{OXM}{OpenFlow Extended Match}
\nomenclature{PBB}{Provider Backbone Bridge Protocol}
\nomenclature{QoS}{Quality of Service}
\nomenclature{RAM}{Random Access Memory}
\nomenclature{RTT}{Round Trip Time}
\nomenclature{SDN}{Software Defined Networking} 
\nomenclature{SCTP}{Streaming Control Transfer Protocol}
\nomenclature{TCP}{Transmission Control Protocol}
\nomenclature{TDD}{Test Driven Development}
\nomenclature{TLS}{Transport Layer Security}
\nomenclature{TLV}{Type-Lenght-Value}
\nomenclature{TTL}{Time to live}
\nomenclature{XML}{Extensible Markup Language}
\printnomenclature
\cleardoublepage
% ---

\pagenumbering{arabic}

% ---- ELEMENTOS TEXTUAIS ----
\textual

% ---- Introdu\c{c}\~{a}o ----


% ---- Cap\'{\i}tulos ----
\chapter{Introduction}
\label{cap:intro}

Traditional computer networks have been successful in their most basic goal: making packets originated from a source location reach a destination \cite{Shenker95fundamentaldesign}. However, the exponential growth in Internet users, emerging use cases and applications have become a challenge for network carriers and administrators. These professionals should be able to handle and to master more and more complex scenarios and configurations \cite{Feldmann:2007:ICD:1273445.1273453}. Furthermore, network equipments are strictly closed or only offer a small set of options for users who want to add their own functionalities and applications. Consequently, innovation in computer networks is compromised, compelling companies to wait for new features on software updates or, worse, to buy a new network box. 

In order to address these issues, inspired by older technologies that have followed similar concepts and evolved \cite{Feamster:2014:RSI:2602204.2602219}, a new network model was needed. The Software Defined Networking (SDN) \cite{DBLP:journals/corr/KreutzRVRAU14}, is a network architecture in which the control plane of network switches is decoupled from the forwarding plane, as illustrated by Figure 1. The control plane is responsible for the management of one or more elements from the forwarding plane. The applications running on top of the control plane can program the data plane to execute determined actions according to the packet type received by an equipment or some network event.  As a result of this flexibility to control the forwarding plane, network equipments may receive new functions and do not need to be replaced when the need for a new functionality arises. Moreover, the network resources can be fully exploited by some smart resource allocation, like network virtualization \cite{FLOWVISOR} \cite{Al-Shabibi:2014:OMY:2620728.2620741}, leveraging the network to its full potential.

\begin{figure}[h!]
\centering
\includegraphics[width=\textwidth,keepaspectratio]{cap1/TraditionalvsSDN.pdf}
\caption{Traditional and SDN models}
\label{fig:traditional_vs_sdn}
\end{figure}
    
The first and the most common standard southbound interface for SDN is the OpenFlow protocol \cite{McKeown:2008:OEI:1355734.1355746} \cite{2012onf_sdn}. The OpenFlow specification describes the interaction between an OpenFlow compliant switch and an OpenFlow controller. Basically the controller installs flow entries for the   switch flow table, so the switches may apply instructions based on the network traffic.


\section{Motivation}
\label{sec:sec01}

Among the reasons for the fast evolution of the OpenFlow protocol is the experimental work led by researchers from Standford, where the protocol was born. New features and capabilities were validated on a software switch implementation, allowing researchers to create and to try new control plane applications. Soon, advances and emerging use cases took the industry attention for SDN and OpenFlow. This culminated in the creation of the Open Network Foundation (ONF), an organization composed by big network players and vendors, as by emerging startups. This organization became responsible for new OpenFlow versions since the version 1.2 and started working on new and enhanced features, resulting in the version 1.3 less than one year after 1.2. 

On the other hand, OpenFlow controllers and switches did not follow the protocol advancements, notwithstanding the fast evolution, resulting in lack of alternatives to experiment new capabilities and anticipation of new applications that could benefit from new features.

With this scenario in mind we found the emerging need to upgrade these tools, allowing fast experimentation and validation. By keeping up the pace with OpenFlow new versions, we expect to contribute to the future of the protocol, driving companies and researchers to develop applications in the state of the art and  enabling future OpenFlow version to be build and tested upon our work.  

\section{Objectives}
\label{sec:sec02}

The objective of this work is the development of a programmable OpenFlow 1.3 software switch to enable fast, real and flexible experimentation for SDN research and education on OpenFlow networks. To achieve this, the software switch must meet the following requirements:  

\begin{enumerate}

\item  \textbf{OpenFlow protocol feature completeness.} All required and optional features shall be implemented, allowing a full OpenFlow experience, without limitations for SDN researchers and developers.   

\item  \textbf{Code simple, easy to prototype and extend.} The code must be simple enough to be modified by anyone with a basic level of programming and understanding of OpenFlow. For this reason, easy insertion of features should be favored in lieu of performance. This requisite meets research needs that goes beyond the OpenFlow specification (e.g: the addition of new messages, new algorithms for group processing, changes to the pipeline, etc). Also, it encourages and helps users to search and to fix bugs quickly, preventing work interruption while waiting for an official patch to correct the switch.       

\item  \textbf{Straight forward integration with experimentation environments and emulation tools.} The switch must integrate with both real and emulated environments, ensuring seamless communication with other switches and controllers, without great modifications. Minor changes are acceptable due to specific platforms requirements: for instance, different processor architectures.

\item \textbf{Maximum throughput equal or higher than 100 Mbps.} High performance is not one of the project goals. However, there are features that play with the switch packet rate. Therefore, for a significant user experience, the switch must be able to support rates of at least 100 Mbps. This value was estimated based by the minimum bandwidth required by common Internet applications. Table \ref{tab:appband} is a mix from values obtained by \cite{Chen:2004:QRN:1234242.1234243} and some popular Internet services. Considering the applications and the bandwidth usage, we found that 100 Mbps is a value to perform a reasonable number of different experiments.          

\begin{savenotes}
    \begin{table}[h]
    \centering
    \caption{Minimum bandwidth requirements for common internet applications}
    \label{tab:appband}
    \begin{tabular}{|l|l|l|}
    \hline
    \textbf{Application}          & \textbf{Bandwidth (Mbps)}                         \\ \hline
    Web Browsing                      & 0.038                                        \\ \hline
    Email                             & 0.01                                          \\ \hline
    Telnet                            & < 0.001                                          \\ \hline
    Audio Broadcasting                & 0.08 to 0.375\footnote{Spotify - https://support.spotify.com/us/learn-more/faq/\#!/article/What-bitrate-does-Spotify-use-for-streaming}                                          \\ \hline
    Video Broadcasting                & 0.5 to 60 \footnote{Netflix - https://help.netflix.com/en/node/306}                                          \\ \hline
    
    \end{tabular}
    \end{table}
\end{savenotes}

\end{enumerate}

\pagebreak
\section{Text Structure}
\label{sec:sec03}

In this Introduction we explained the motivational aspects that justify this work. Also, we give a clear explanation for the objectives of this project. 

In Chapter~\ref{cap:cap02} we present a Literature Review. Related OpenFlow software switches' current functionalities are discussed from the point of view of our implementation requisites. Furthermore, we introduce other tools which are important parts of the OpenFlow ecosystem. 

In Chapter~\ref{cap:cap03} we take a look at the architecture of the software switch which is compliant with OpenFlow 1.3. We explain the modules relationship and roles within the OpenFlow pipeline.

In Chapter~\ref{cap:cap04} we highlight implementation details of OpenFlow 1.3 features in our architecture.  

In Chapter~\ref{cap:cap05} we evaluate the software switch in terms of common OpenFlow benchmarks and compare with related work.

Finally, in Chapter~\ref{cap:conclusion} we give our conclusion remarks. This chapter highlights results, presents known use cases and discusses possible improvements in future works. 
\chapter{Literature Review}
\label{cap:cap02}

OpenFlow is the central theme of this project and deep understanding of the protocol is required. Therefore, the OpenFlow 1.3 specification \cite{ofspec13} is the main document of our bibliographic base. We also have studied and compared available implementations of other OpenFlow software switches. Moreover, some tools to evaluate our work are worth mentioning, because of the necessary adaptations made to overcome the lack of support for the most recent OpenFlow versions. We investigated controllers, test frameworks and packet dissectors, all candidates to compose our OpenFlow test environment. The next sections will give an overview about OpenFlow and relevant tools related to this work. 

\section{OpenFlow}
\label{sec:sec21}

OpenFlow is an open standard communication interface between switches and controllers, allowing centralized control and programmability in the network. The basic OpenFlow switch is composed by one or more flow tables, a group table and one or more OpenFlow channels for communication with OpenFlow controllers. Figure \ref{fig:logicalswitch} is a logical view of the minimal elements required by a switch. 

\begin{figure}[h!]
\centering
\includegraphics[height=8cm,width=\textwidth,keepaspectratio]{cap2/OFSwitch.pdf}
\caption{OpenFlow switch minimal elements.}
\label{fig:logicalswitch}
\end{figure}
\pagebreak

OpenFlow controllers can install flows into the switch flow table. A flow consists of match fields, counters and instructions applied to matching packets. A packet matches a flow if the protocol header fields' values are the same as those specified in flow match fields. The most recent version requires 13 match fields, shown by Table \ref{tab:OFRequired}, but has optional support for more than 30 protocols fields from layers 2, 3 and 4 of TCP/IP network stack.

The OpenFlow pipeline starts at the first flow table and continues to additional tables. Flows are matched in order of priority and the associated instructions are executed. Only two instructions are required for OpenFlow switches: Write-Actions, in which actions are executed at the end of the pipeline, and Goto-Table, to jump to tables with an id greater than the table where the instruction is executed. Optional instructions are Meter, to direct the packet to some meter for QoS, Apply-Actions, for immediate action application, Clear-Actions, to clear all actions written by a Write-Actions instruction, and Write-Metadata, to write metadata information to be carried across the tables. 

\begin{table}[h!]
\caption{OpenFlow required match fields}\label{tab:OFRequired}
\centering
\begin{tabular}{|l|l|}
\hline
\textbf{Field} & \textbf{Description}                \\ \hline
Inport         & Ingress Port                        \\ \hline
Eth Dst        & Ethernet destination address         \\ \hline
Eth Src        & Ethernet source address              \\ \hline
Eth Type       & Type of the packet, after VLAN tags \\ \hline
IP Proto       & IPv4 or IPv6 next protocol number   \\ \hline
IPv4 Src       & IPv4 source address                 \\ \hline
IPv4 Dst       & IPv4 destination address             \\ \hline
IPv6 Src       & IPv6 source address                 \\ \hline
IPv6 Dst       & IPv6 destination address            \\ \hline
TCP Src        & TCP source port                     \\ \hline
TCP Dst        & TCP destination port                \\ \hline
UDP Src        & UDP source port                     \\ \hline
UDP Dst        & UDP destination port                \\ \hline
\end{tabular}
\label{my-label}
\end{table}

Actions can perform modifications on packets, discard or send them to the group table or simply output to some specific port. The only required actions are Output, to send the packet through a port, and Drop. The packet modification actions are all optional, but their implementation is recommended, as they give more power and options to OpenFlow networks. The optional actions are Group, to process the packet through a specific group; Push-Tag/Pop Tag, for addition and deletion of VLAN, MPLS and PBB tags; Set-Field, to modify packets header fields; Change TTL, an action to modify MPLS and IP TLL; Set-Queue, to determine which queue is attached to a port and will be used for scheduling in packet forwarding.

OpenFlow groups are a way to perform more complex forwarding actions. When a packet is sent to a group it is cloned and executed by sets of action buckets. This abstraction enables flooding, multipath, link aggregation and other techniques that demand transmission of packets through more than one port. 

The last essential block is the OpenFlow channel. The main connection between controller and switch is done by one of the following transport protocols: TCP or TLS, where the second is recommended because it enables data encryption of the transmitted data. Auxiliary connections are also allowed and it is possible to have UDP connections for transmission of less sensitive OpenFlow messages.

An optional element of the OpenFlow switch is the Meter Table. This table comprises different types of Meter Bands, which have a speed limit and apply a determined QoS action in the case of a packet flow exceeding the determined limit. There are two types of bands covered by the OpenFlow 1.3 specification: Drop, to discard packets; and DSCP Remark, to decrease the drop precedence of the DSCP field of the IP header.

\subsection{One Day in the Life of An OpenFlow 1.3 switch in 10 steps}

To illustrate the operation of an OpenFlow switch we will give a common and simple example. A learning switch is a layer 2 network equipment that learns the port to which a host is connected. Learning happens when a packet from a host arrives at the switch for the first time. The switch then obtains and stores the host Media Access Control (MAC) address associated with the port number. Next time, when another host sends a packet to the previous learned address, the switch will forward it directly, instead of flooding to all ports. 

In legacy devices, the control software is embedded into the switch hardware and the MAC addresses are stored in a Content Addressable Memory (CAM) table. In an OpenFlow scenario, the learning happens inside the controller and the forwarding rules are stored in the Flow Table. We will describe the steps of learning and forwarding processes of an OpenFlow switch controlled by a simple learning switch application, considering the topology in Figure \ref{fig:simpletopo}. 

\begin{figure}[h!]
\centering
\includegraphics[height=5cm,width=\textwidth,keepaspectratio]{cap2/SimpleTopology.pdf}
\caption{Simple topology for the learning switch example.}
\label{fig:simpletopo}
\end{figure}

In the initial state the switch Flow table is empty, \textit{Host 1} and \textit{Host 2} do not know anything about each other and the  \textit{Controller} is about to connect with the \textit{Switch}. 

\begin{enumerate}

\item The \textit{Controller} establishes a connection with the \textit{Switch}. As packets sent to an empty Flow Table are dropped according to the OpenFlow 1.3 specification, the  \textit{Controller} installs a low priority flow to direct every non matching packet to him. Table \ref{tab:initialtable} shows the Flow table state after the installation of the first flow.  

\begin{table}[h]
\centering
\caption{Switch Flow Table state after controller connection}
\label{tab:initialtable}
\begin{tabular}{|l|l|l|}
\hline
\textbf{Match}  & \textbf{Priority} & \textbf{Instruction}                              \\ \hline
all             & 0                 & apply actions -> output:controller                \\ \hline
\end{tabular}
\end{table}

\item \textit{Host 1} wants to transmit a file to \textit{Host 2}. As it does not know about \textit{Host 2} MAC address, it sends an ARP \cite{rfc826} request to the network.

\item The ARP request packet enters the \textit{Switch} and matches the unique flow installed in the Flow Table. The action is applied, sending the packet to the  \textit{Controller} in an OpenFlow \textit{Packet In} message.

\item The  \textit{Controller} receives the \textit{Packet In}. The message contains information about the packet input port and headers. The  \textit{Controller} application learns and stores the \textit{Host 1} input port and MAC address. After that, it sends a \textit{Packet Out} message back to the \textit{Switch}, with an action to flood the packet in every \textit{Switch} port.   

\item The \textit{Packet Out} message arrives at the \textit{Switch} and the packet is flooded to every port, except the port where it came from.

\item The ARP request is delivered to \textit{Host 2}. Then, an ARP reply is sent back with the \textit{Host 2} MAC address required by \textit{Host 1}.

\item The ARP reply arrives at the \textit{Switch}. It matches the only present flow and is sent in a \textit{Packet In} message to the  \textit{Controller}.   

\item Now, the  \textit{Controller} checks if the packet MAC address is known. As the address is not present, it stores the input port and the MAC. Now it checks if the destination address was stored previously. The destination to \textit{Host 1} is already known. The  \textit{Controller} installs a new flow into the \textit{Switch} Flow Table. The flow illustrated in the second row of Table \ref{tab:secondtable} matches every packet destined to \textit{Host 1} MAC address and outputs the packet into port 1. After installing the flow, it sends the ARP reply to the \textit{Switch}, encapsulated in a \textit{Packet Out} message.

\begin{table}[h]
\centering
\caption{Switch Flow Table state after learning Host 1 address}
\label{tab:secondtable}
\begin{tabular}{|l|l|l|}
\hline
\textbf{Match}                 & \textbf{Priority}   & \textbf{Instruction}                              \\ \hline
all                            & 0                   & apply actions -> output:controller                \\ \hline
eth dst:HOST 1 MAC             & 100                 & apply actions -> output:1                         \\ \hline
\end{tabular}
\end{table}

\item This time The ARP reply is not flooded to all ports. As the  \textit{Controller} knew about the destination, it is sent directly to \textit{Host 1}. After receiving the ARP reply, it starts transmitting the file to \textit{Host 2}.

\item Again, the first packet of the file transference to \textit{Host 2} is sent to the  \textit{Controller}. Now it knows about \textit{Host 2} MAC and origin port. It installs another flow, shown in the third row of Table \ref{tab:finaltable}, and it sends the \textit{Packet Out} message to the \textit{Switch}. The \textit{Switch} sends the packet directly to \textit{Host 2}. From now on, the packets are not  sent for the \textit{Controller} anymore because the installed flows are matched. 

\begin{table}[h]
\centering
\caption{Switch Flow Table final state}
\label{tab:finaltable}
\begin{tabular}{|l|l|l|}
\hline
\textbf{Match}                 & \textbf{Priority}   & \textbf{Instruction}                              \\ \hline
all                            & 0                   & apply actions -> output:controller                \\ \hline
eth dst:HOST 1 MAC             & 100                 & apply actions -> output:1                         \\ \hline
eth dst:HOST 2 MAC             & 100                 & apply actions -> output:2                         \\ \hline
\end{tabular}
\end{table}


\end{enumerate}        

\section{OpenFlow software switches}
\label{sec:sec22}

OpenFlow software switches play an important role in the protocol evolution. At first, they were a low cost solution to create and experiment your own SDN applications in a controlled and smaller environment, before the deployment in a production environment. With the new SDN approaches for network virtualization \cite{Tseng:2011:NVC:2117686.2118540} \cite{Drutskoy_scalablenetwork}, software switches have been receiving a lot of attention from the industry \cite{NSX} and academy \cite{DBLP:confcloudnetEmmerichRWC14}. SDN virtual switches interconnect virtual machines from data center tenants and help scaling a plethora of traffic engineering applications. For instance, tenant isolation is easily achieved using an OpenFlow software switch to connect virtual machines. Usually, it is implemented using VLAN segregation, which is not scalable for more than 4096 tenants (the total number of possible VLAN identifiers). In turn, OpenFlow switches offer more granular options to segregate the network hosts, due to the number of fields available for flow matching, which eliminates the need to segregate hosts using only a layer 2 domain.

\begin{table}[h]
\caption{Comparison of OpenFlow software switches}
\label{tab:relatedswitches}
\begin{tabular}{|l|l|l|l|l|}
\hline
\textbf{Switch} & \textbf{Language} & \textbf{Emulation tool integration} & \textbf{Mode}    & \textbf{OF-Config} \\ \hline
Reference       & C                 & Yes                        & userspace        & No                         \\ \hline
Open vSwitch    & C                 & Yes                        & userspace/kernel & No                         \\ \hline
LINC            & Erlang            & No                    & userspace        & Yes                        \\ \hline
Trema           & C/Ruby            & No                    & userspace        & No                         \\ \hline
\end{tabular}
\end{table}

Before the implementation, we investigated four OpenFlow software switches that supported OpenFlow 1.0, since there was no sense in implementing basic OpenFlow features from scratch. Table \ref{tab:relatedswitches} is a comparison of the examined software switches and reflects their state in the year of 2012. The columns are related to the objectives defined in chapter \ref{cap:intro}. Popular programming languages have the power to reach a bigger audience. The integration with emulation tools eases experimentation and speeds up testing. The mode column is related to switch performance, since kernel space implementations tend to be more efficient than an user space implementation. In a kernel implementation, packets are processed directly in the kernel space, eliminating the overhead of traversing packets between the kernel and user space.

In the mean time of this work, most of the analysed software switch projects started to add support for newer OpenFlow versions. Figure \ref{fig:ofswitchtimeline} shows the switches' timeline for the implementation of most recent OpenFlow versions. Colored intervals mean that these OpenFlow versions are still in progress or do not include all features described on the specification, as we will show in section \ref{sec:FeatureComplete}.

\begin{figure}[H]
\centering
\includegraphics[height=6cm,width=\textwidth,keepaspectratio]{cap2/SoftSwitchTimeline.pdf}
\caption{OpenFlow Software Switches: version support timeline.}
\label{fig:ofswitchtimeline}
\end{figure}

The next subsections will provide a short individual description of the software switches investigated: OpenFlow reference switch, Open vSwitch, LINC and Trema. 

    \subsection{OpenFlow reference switch}
    \label{sec:sec221}
    
     The first OpenFlow switch is known as reference switch because it was implemented by Stanford researchers directly involved with the OpenFlow protocol creation and the first specification releases. The code is written using the C language and its simplicity is one of the reasons for the implementation to other platforms. Two important ports are: NetFPGA boards \cite{netpfgaof}, eliminating the disadvantage of the user space implementation; and OpenWrt \cite{pantou} for wireless routers. These efforts enabled low cost options to test OpenFlow on real hardware during earlier protocol stages.    
     
     The last OpenFlow version implemented by Stanford researchers was 1.0, but after the version 1.1 release there was an update, based on the reference switch, released by Ericsson Traffic Lab, called of11softswitch \cite{of11softswitch}. To conform changes such as multiple tables, the switch forwarding plane was rewritten, but still retained all of the base software switch characteristics.     

     \subsection{Open vSwitch}
    
    Considered the \texttt{de facto} switch for virtual networks, Open vSwitch \cite{Pfaff_e.a.:extending} (OVS) is a mature and constantly evolving open source project. The efficiency provided by the kernel module and the functionalities beyond OpenFlow turns OVS into a great solution to replace the original Linux bridge. In addition to the basic connectivity provided by traditional bridges, OVS offers a flexible option to manage and program the packet forwarding behavior. OVS uses another protocol for switch management. The Open vSwitch Database Management Protocol (OVSDB) is one of the most praised OVS features and is designed to manage all running switch instances, permitting the control of distributed virtual network nodes. With OVSDB, a network engineer can create, configure and delete OVS ports and tunnels from a centralized location.

    \subsection{LINC}
    
    LINC \cite{linc} is an userspace software switch written in the Erlang programming language and has different support levels, considering the number of working features, for OpenFlow versions from 1.2 to 1.4. The main advantage of this switch is the support to OF-CONFIG \cite{ofconfig}, the OpenFlow switch configuration protocol. As an userspace implementation, efficiency is not one of its strong points. However, it promises flexibility, fast development and testing of new OpenFlow features. 
    The Erlang language is not a disadvantage \textit{per se}, but it can be considered a blocking point for developers who want to add their own features. Erlang developers are growing, but their number are still very far from languages like C and Java.

    \subsection{Trema}
    
    Trema is a name most known by the OpenFlow controller, but it goes beyond. It offers a full OpenFlow framework with the tools needed to develop OpenFlow applications \cite{trema}, including an OpenFlow software switch. Also, the framework has its own emulation tool for OpenFlow networks and end-hosts. The main repository of this project features switch with only OpenFlow 1.0. However, there is a repository known as trema-edge where the work for OpenFlow 1.3 is on progress.

\section{OpenFlow Controllers}

    Controllers are considered the brain of an OpenFlow network. Every configuration and forwarding rules are defined by applications running on top of a controller. They are sent in form of OpenFlow messages, which have to conform with the format defined by the specification. For this reason, we need to validate our work testing interoperability between the software switch and a compliant controller. We reviewed the main open source projects, looking for an OpenFlow 1.3 controller or an easy alternative to implement some level of support for OpenFlow 1.3, required for our tests. 

    \subsection{NOX}

    One of the first open source OpenFlow controllers, NOX \cite{nox} was very famous during the first years of OpenFlow. Its popularity was due to a combination of factors, from the C++ implementation and a Python binding, which helped to speed up the prototyping, to a quite simple interface and a good number of example applications. The last official release supported OpenFlow 1.0 version. After some enhancements on the controller speed \cite{nox-mt}, no efforts were made to upgrade it for newer OpenFlow versions. 
    
    \subsection{POX}
    
    Pox is a controller implemented in Python and can be considered a NOX sibling \cite{pox}, created to address the lack of speed when prototyping with NOX. Its main goal is to become the archetypal of a modern SDN controller, featuring some desired SDN capabilities like debugging, new programming models and network virtualization. 
    
    Designed with research in mind, under constant development and a typical controller used for SDN education, POX was a great candidate to be the controller for testing the software switch, but the OpenFlow support still doesn't surpass version 1.0
  
    \subsection{Floodlight}

    The Floodlight controller is a popular OpenFlow controller backed by Big Switch Networks, one of the most prominent SDN startups \cite{floodlight}. It is developed in Java and was designed for high performance and is the core of Big Switch Networks' commercial solution. One of the greatest features of Floodlight is the module loading system that makes it very extensible, allowing it to enable and disable applications at run time. Another great feature is the Open Stack integration, a cloud orchestration platform. Floodlight instances control the virtual switches linking virtual machines orchestrated by Open Stack. Regarding OpenFlow version support, it currently supports OpenFlow 1.0 and 1.3. 
  
    \subsection{Ryu}
    
    Ryu is considered an SDN network framework \cite{ryu}, an abstraction that provides code software components, with generic functionality, easing and speeding development of SDN applications. It is implemented in Python, like POX, but has a different architecture. Designed as software components, developers can create OpenFlow applications like modules. Furthermore, the controller also supports management protocols like OF-Config and Netconf.
    
    Very well documented, with a large number of examples and featuring all OpenFlow versions, Ryu is currently a great choice to test the software switch, but when we started this work OpenFlow support was limited to OpenFlow 1.0. OpenFlow 1.3 support was only available in the middle of our switch implementation.
 

\section{OpenFlow test and emulation}
\label{sec:testemulation}
    The last pieces of a minimal OpenFlow environment are  test frameworks and emulation tools. For our development we are more interested in the first, though a modular emulation software compatible with our software switch may benefit users looking for a complete, controlled and easy to setup testbed. 
    
    \subsection{OF-Test}
    
    OF-Test was the first OpenFlow test framework. Developed by the same team working on Floodlight, OF-Test \cite{oftest} tests basic functionality for OpenFlow 1.0 and 1.1, with 1.2 and 1.3 currently in development. The simpler architecture and Python implementation turns it into a very easy and fast platform to create and run tests. It works connecting the OF-test server to the switch control and the data plane. The server is responsible for monitoring OpenFlow messages and packets sent through and along the planes. If these messages and packets are according to expected results, the test returns OK, otherwise, a failure is reported.
    
    \subsection{Ryu Test Framework}
    
    This test framework is part of the Ryu controller and implements tests to cover all, required and optional, OpenFlow 1.3 and 1.4, actions, instructions and match fields, with a more comprehensive test than OF-Test. The test cases are written in JSON, so there is no need for coding to create a test, which enhances the speed of test creation. 
    
    There is an online certification which continuously tests OpenFlow software and hardware switches, including our work \cite{ryucert}. Results from this certification will be presented in chapter \ref{cap:cap05}.

    \subsection{OpenFlow packet dissectors}

    Packet dissectors are important tools to test if message packets are correct or to check if an specific packet was sent or modified by an OpenFlow output and set-field actions, respectively. Wireshark \cite{wireof} is the most famous program to dissect packets. Although a wide range of protocols are officially supported, OpenFlow support started as an  unofficial  plugin for OpenFlow 1.0 and 1.1. For a long time this plugin was the only option for analyzing OpenFlow traffic. Recently, due to the growth in the number of users requesting for official support, OpenFlow is now developed in the dissector's main repository \cite{wireof}.    
    \subsection{Mininet}
    
   Mininet \cite{Lantz:2010:NLR:1868447.1868466} is a tool for network emulation. In a single machine it runs switches, links and hosts just like a complete virtual network. It is possible to log into the hosts and use programs like Iperf \cite{iperf} and Ping, to measure throughput and check connectivity; specify link parameters as speed and delay; and instantiate a network topology composed of software switches. The capacity to create and to destroy virtual networks allows easy and fast experimentation. 
  
    
\chapter{Architecture}
\label{cap:cap03}

Architectural design is strongly tied down by OpenFlow 1.3 required and optional elements. With the architecture description, we do not intend to repeat the specification. Thus, elements described in the next sections are an specific and conceptual point of view of each software switch component. 

The OpenFlow switch implementation is a fork from the OpenFlow 1.1 reference switch. We choose this implementation as a starting point for our work because of the simpler code base, when compared to other switch implementations discussed in section \ref{cap:cap02}. Although simple, the only documentation available to understand the reference switch was the OpenFlow specification. For this reason, the definition of the architectural design started from the extraction of the previous software architecture. 

Firstly, through a simple reverse software engineering process \cite{Sommerville:2001:SE:375369}, we analyzed the code and listed the switch core components. Next, we identified missing components from the OpenFlow 1.3 specification. After these steps, we found that block structures suggest the application of a bottom-up design \cite{vonMayrhauser:1990:SEM:79005}. In this approach, the basic set of foundational modules and their interrelationships are the foundation for the final architecture. Following these concepts, we came up with the final design for the OpenFlow 1.3 software switch.

In the Figure \ref{fig:switcharq} we show the software switch architecture. The most important block is the Datapath. It consists of OpenFlow internal elements such as Flow, Meter and Group Tables, and a Packet Parser as well. The other three blocks operate on different levels along with the Datapath. From the top level, the blocks Datapath, Marshaling/Unmarshaling library and Communication Channel are part of the OpenFlow message layer. Below, Ports and Datapath form the network packets layer, where packets arrive, are processed and usually sent back for the network.\footnote{Some instructions, actions and even an empty table may cause a packet drop in the Datapath.} Except for the special case of the \textit{Packet In} message, in which packets can be sent for the controller, these two layers do not interact. In Figure \ref{fig:switcharq}, dotted lines illustrate some possible paths a network packet can travel between the Ports and the Datapath. Solid lines denote the OpenFlow messages traveling in the OpenFlow message layer. Arrows mean the direction packets and OpenFlow messages can take across the switch components. In this chapter, we present each software switch component individually, detailing each block roles and interactions with other elements.

\begin{figure}[H]
\centering
\includegraphics[height=12cm,width=\textwidth,keepaspectratio]{cap3/switcharq.pdf}
\caption{Software switch architecture}
\label{fig:switcharq}
\end{figure}

	\section{Ports}

	OpenFlow ports are the entrance and exit doors for  network packets of the OpenFlow switch pipeline. A software switch instance running on a machine may use their physical or virtual interfaces as port elements. Physical port elements can take control over Ethernet or WiFi interfaces, allowing the creation of real network topologies. Although limited by the speed of the software switch, the possibility to create a low cost testbed enriches the experience of users developing and testing OpenFlow applications.     

	Ports functions on the switch are not limited to the task of send and receive packets. There is a set of responsibilities associated with the OpenFlow protocol and the pipeline. Functionalities are:

	\begin{itemize}

	\item OpenFlow enables some level of control over a port behavior. A port modification message permits the configuration of the port state. Ports can be administratively set to drop all received or forwarded packets, forbid the generation of \textit{Packet In} messages from arriving packets, and brought down. Ports elements should handle these messages and change the port behavior according to the configuration sent.

	\item OpenFlow Ports keep the current state of the physical link. This information is not configurable by an OpenFlow controller, but the switch should inform the control plane about link state changes. Ports monitor the state of the port link and update the information according to changes.

	\item Packets encapsulated in \textit{Packet In} usually have only the header sent within the message. A buffer stores the packet, while waiting for the controller decision after the \textit{Packet In}. Ports elements store these packets and resend them for further processing.

	\item An OpenFlow controller can ask the switch about a port description. The software switch element retrieves information such as current and max operating speeds from the interfaces of the machine and stores it. On a port description request, the element handles the message and sends the required information to the control plane.

	\item Queues creation are not part of the OpenFlow protocol. However, OpenFlow can configure port queues, created by whichever mechanism, to be associated with a switch port. Ports are responsible for handling queue association and configuration.  

	\item Ports must update port and queue packet counters.           
	\end{itemize}

	\section{Packet Parser}

	Before entering the software switch OpenFlow pipeline, packet protocol fields are extracted by the Packet Parser element. Parsing packets was a formally defined task until OpenFlow 1.1. The main reason to define how packets should be parsed is to guarantee parsing consistency, but it limits switch designers and demands algorithm updates for each new protocol addition. For this reason, further specifications removed how packets should be parsed and match fields are now defined only logically.

	A Packet Parser element converts extracted protocol fields of a packet to an internal flow entry format. Two scenarios may trigger this function:   

	\begin{itemize}

	\item A network packet enters the switch through one of its ports.    

	\item  If the packet was modified by an action and is resubmitted for the pipeline, or sent to a table ahead by a \textit{Go To Table} instruction, packet revalidation is required. Hence the packet is processed by the Packet Parser again. In Figure \ref{fig:switcharq}, after passing by the Flow Tables, there is an arrow representing packet return to the Packet Parser.        

	\end{itemize}

	Further OpenFlow extensions, supporting new protocols, directly affect the packet parsing. Modifications are required in order to add new match fields for the Packet Parser. Therefore, a flexible and extensible Packet Parser element is desirable.  

	\section{Flow Tables}

	Flow Tables are the heart of an OpenFlow switch architecture. They are the elements where flow entries are stored and the OpenFlow pipeline starts. Although the use of multiple Flow Tables is optional - the specification mandates at least one table - its implementation is recommended, as even simple applications can not scale in switches with only one Flow Table \cite{tableExplosion}.  
	
	Flow Tables roles in the software switch are listed below.

	\begin{itemize}

    \item In case of nonexistence of a table miss flow entry, Flow Tables have to implement some default action for not matched packets. Currently, the default action is drop the packets. 

	\item Handle \textit{Flow Mod} messages sent by the controller. These messages may add or delete flow entries, or change the instruction set from currently installed flows.  

	\item Flow Tables must be able to have their capabilities reconfigured by a controller. These table features can express the table supported properties. The instructions' type and the match fields allowed in the table are examples of properties. Also, some fields show relevant information for an OpenFlow application. For instance, the table identifier value is an information required to add a new flow, and the max number of flow entries should be considered to avoid scalability problems.           

    \item A Packet look up must be performed upon the receiving of a packet. The operation looks for a Flow Table entry that matches the packet. In the case of a match, the switch executes the instruction set associated with the flow entry. This is the most common activity in Flow Tables.   
    
    \item Keep table statistics about the number of active flow entries, number of look ups and matched packets.  

	\end{itemize}

	\section{Group Table}

	Group Table empowers OpenFlow forwarding options. Packets reach the Group Table after matching a flow entry containing a group action, in one of the Flow Tables. 
	
	Group entries are stored into the Group Table. Each group entry contains an identifier, a type, counters and action buckets. Action buckets are an ordered list of action sets to be executed according to the group type. Figure \ref{fig:grouptable} represents a group table filled with groups of All, Indirect and Fast Failover types. The layout of the specific group types is important, because it defines Group Table attributions, as shown by the responsibilities listed here.
	
	\begin{figure}[H]
    \centering
    \includegraphics[height=7cm,width=\textwidth,keepaspectratio]{cap3/GroupTable.pdf}
    \caption{Group Table internals}
    \label{fig:grouptable}
    \end{figure}

	\begin{itemize}
	
	\item The Group Table have to guarantee group type restrictions. For instance, indirect groups support only one action bucket. 

	\item The Group Table must handle modification messages and perform consistency checks in the case of group chaining. Chained groups point to other groups and may cause loops that should be avoided by the element.  

	\item Fast failover groups require monitoring switch ports and group buckets for state changes. For this reason the Group Table is responsible for checking bucket liveness when choosing the first live bucket.

	\item A Group Table that supports the select group type has to implement a schedule discipline algorithm to choose which bucket will be applied to the packet.

	\end{itemize}

	\section{Meter Table}
	\label{sec:MeterTable}

    The Meter Table is an element to perform simple QoS operations. Per-flow meters are attached to flow entries through the \textit{Meter} instruction. A meter entry is composed by a meter id, counters and meter bands. The QoS operations to apply are defined by the meter bands. A meter band must have a type and rate value, which is the boundary to apply the action determined by the type. Figure \ref{fig:metertable} illustrates the internals of a Meter Table, with two meter band types.   

    \begin{figure}[H]
    \centering
    \includegraphics[height=7cm,width=\textwidth,keepaspectratio]{cap3/MeterTable.pdf}
    \caption{Meter Table internals}
    \label{fig:metertable}
    \end{figure}

    Meter Table responsibilities include:

    \begin{itemize}
    
    \item Creation, destruction and modification of meter entries.
    
    \item Matched packets, from flows pointing to the Meter Table, rate measurement. 
    
    \item Keep and update counters for statistics of packets processed by an entry.
    
    \item Process the packets according to band operation. A Drop meter band type discards the packets and the DSCP remark changes the IP packet drop precedence. 
    
    \end{itemize}

    \section{Marshaling/Unmarshaling library}
    \label{(un)pack}
    OpenFlow messages defined by the specification follow a proper format for transmission in the network. Messages are 8-byte aligned, so there may be insertion of padding fields to follow this alignment rule. Another requirement for the message format is the byte order. The preferred format for packets sent through the network is the network byte order \cite{rfc1700}. As OpenFlow messages are sent over IP networks, their messages should be assembled following the Big-Endian format.
    
    The architectures of machine's processor may operate on different byte-endianness. For instance, Intel processors use the Little-Endian byte order \cite{little-endian}. So, in order to handle and assemble OpenFlow messages, conversion is required for non Big-Endian architectures.     
    
    For the mentioned reasons, a library which abstracts byte-endianness and adds any required bytes to ensure right message format is required. A Marshaling/Unmarshaling library is not an element defined by OpenFlow specification. Its main function is the translation of OpenFlow messages from the network format to an internal format and vice-versa. The library responsibilities are the following:
    
    \begin{itemize}
    
    \item Every OpenFlow message must have a function that packs and unpacks it. Pack is the function which converts internal structures into network format. While unpack turns received messages into an internal structure.
    
        \subitem - When packing, the library has to add any necessary padding bytes.
        \subitem - On packing, the message should be assembled in network byte-order.
        \subitem - On unpacking, the library must translate the message fields to the switch host architecture byte-order.
    
    \item Some handling of OpenFlow message errors is done in this level. The library must raise errors for messages with wrong length or bad arguments. 
    
    \end{itemize}

    \section{Communication Channel}	

    The OpenFlow software switch communicates with controllers through the Communication Channel. This element connects with the Datapath and the controller and acts as a proxy between them. This element exists because the implementation of the communication channel is not defined by the specification. Since the message format is respected, implementations are free to choose the connection protocol. For instance, when security for the channel is a requirement, a protocol like TLS should be used to encrypt the messages. 
    
    In the software switch, the Communication Channel roles are:
    
    \begin{itemize}
    
    \item The Communication Channel must establish a TCP connection with the switch and the controller.
    
    \item Connection setup is a Communication Channel responsibility. After a TCP connection, the switch negotiates the protocol version with the controller. This process, known as handshake, is managed by the Communication Channel. 
    
    \item The channel may use multiple connections with a single controller at the same time. These connections can be used to send OpenFlow in parallel or to create specific channels for some message types.
    
    \item A Communication Channel is responsible for opening connections to enable switch communication with more than one OpenFlow controller. 
    
    \end{itemize}
    
    
    
\include{chapters/cap4}
\include{chapters/cap5}
\chapter{Conclusion}
\label{cap:conclusion}

Six years ago SDN and OpenFlow caused a stir in the world of computer networks. Although data and control plane separation is not a new idea, the flexibility and programmability enabled by OpenFlow started a wave industry efforts to support the protocol in their products. Several OpenFlow 1.0 switches, controllers and test frameworks emerged from this movement confirming the growing interest in the technology. Large networks operators, such as Google and Facebook, embraced OpenFlow interested on its potential. An organization, named Open Network Foundation, was created to speed up the OpenFlow development and adoption. Quickly, new versions of the protocol were released. This time, however, implementations did not aroused at the same time.

To keep up with the pace of the technology and enable research that leverages the new functionalities, we found the need to implement an OpenFlow 1.3 software switch. More than a full compliant implementation requirements included minimal performance and ease of experimentation. This effort lead to the open source software implementation of the first OpenFlow 1.3 switch. 

Today, the software switch is a well known open source project and a cheap and friendly option to experiment OpenFlow 1.3. Although new software switches supporting OpenFlow 1.3 are now available, this work is still a solid and relevant option to prototype and develop new OpenFlow applications. In the next two sections, we present some relevant obtained results and notorious use cases. Finally, we conclude this chapter discussing future areas for research and improvement in the software switch.     

\section{Results}
\label{sec:results}

In this section we list positive results achieved on the dissemination of our work:  

\begin{itemize}
\item \textbf{Development of an Open source community.} GitHub choice to host the code proved to be a great way to reach a high number of users. Figure \ref{fig:ghstats}, shows information that confirm the tool popularity. In a 14 days the interval, the software switch repository had 3125 accesses, with 796 unique visitors and cloned 97 times. 

With respect to code contributions, there are 15 users listed in GitHub that submitted pull requests. Also, there is a number of contributors that send patches through other means.

This result is very important because the creation of a community around open source code gives visibility, helps to spread the software and speed reports of detected bugs. 

\begin{figure}[H]
\centering
\includegraphics[height=20cm,width=\textwidth,keepaspectratio]{cap5/GH.pdf}
\caption{GitHub statistics.}
\label{fig:ghstats}
\end{figure}

\item \textbf{Part of Mininet installation options.} The software switch is included among the installation options of Mininet. It makes easier for users to start experimenting with OpenFlow 1.3. In addition, the range of people trying     

\item \textbf{Publications.} This work gave origin to three publications. The first paper is about IPv6 support on OpenFlow. It was the first. The second is an invited paper that gives perspectives on SDN for home network. These ideas were inspired by the software switch port to OpenWRT. Finally, the last paper is an overall presentation of this project, highlighting architectural and implementation details. These publications are listed on Annex \ref{AnnexB}.  
\end{itemize}


\section{Use Cases}
\label{sec:cases}

Known use cases show that the software is an important tool in the advance of the state of art on SDN research and development. As there is a large number of projects who are using or have made use of our work, we will list some notorious examples: 

\begin{itemize}
\item \textbf{Base for new OpenFlow features implementation.} The ONF group responsible by the addition of new OpenFlow features decided to publish new features only if properly implemented and tested. As OpenFlow 1.3 basic functionalities is required for 1.4, the Extensions Work Group chose our work as one of the base software switches used for new functionality prototyping \cite{ONFproto}. 

\item \textbf{Academic.} The software switch has found good adoption by the academic community. For instance, works published in renowned conferences \cite{Reitblatt:2013:FDF:2491185.2491187} \cite{Bianchi:2014:OPP:2602204.2602211} and master dissertations \cite{Paris} \cite{ShahmirShourmasti656472} cite our software as the OpenFlow 1.3 switch chosen for their experiments .  

\item \textbf{Industry.} Industrial development is harder to track because this development are usually closed. However, one successful case is in the development of an application for the Open Network Operating System (ONOS). Built by two companies teams, Dell and ON.Lab, the Segment Routing implementation used the software switch for its simplicity \cite{ONOS}.         
\end{itemize}

\section{Future Work}

Each architectural component from the software switch has space for improvements. New algorithms and data structures are objects of study for the Flow Table matching. More complex and precise algorithms for rate limiting might be considered for Meter Table better performance. As for groups, new bucket select types may be a subject for academic research. 

While there are open ideas for further research and development, some optimizations and features are planned for the software switch in the medium-term. These major improvements are listed below:

\begin{itemize}

\item \textbf{Support for OpenFlow 1.4}. OpenFlow 1.4 is an expansion of OpenFlow 1.3 and it would good to keep the pace with the OpenFlow evolution. Some OpenFlow 1.4 features are already implemented, as stated in section \ref{sec:cases}, however we would like to have both versions supported in a single switch running instance, without need to split the code in two different programs.

\item \textbf{Hash based match.} Results found in experiments presented on section \ref{sec:bandflows} show a huge loss in performance due to linear matching. To solve this problem, flows entries might be represented as hash value into the Flow Table. Then, packet fields could also be turned into a hash and looked up in the Flow Table. This would give constant performance for the Flow Table look up. However, there are some questions to answer:

    \begin{itemize}
    \item How to handle flow priority? Since flows should be matched in order of priority, how to ensure the first hash value for a flow is the one with higher priority? 
    \item How to deal with field masking? Some flow match fields may have a mask, so they should be considered in the hash calculation. The question is how to efficiently search and apply these masks to the packet hash calculation. 
    \end{itemize}

The search for an answer for these questions opens space for new research in OpenFlow and SDN, because this questions are not only related to the software switch. 
  
\item \textbf{New packet parsing engine}. The software switch relies on Netbee library to parse packets. While Netbee adds flexibility and extensibility for the parsing and ease the addition of new protocols to OpenFlow, its code is not frequently updated, not following dependencies upgrades. This breaks the software switch compilation in more recent Linux versions, because of more recent versions of libraries required by Netbee. After   Due to the number of compilation issues related to netbee, a new packet parsing module must replace the current third party library.

\end{itemize}


% --- Finaliza a parte no bookmark do PDF, para que se inicie o bookmark na raiz ---
\bookmarksetup{startatroot}%
% ---


% ---- ELEMENTOS P\'{O}S-TEXTUAIS ----
\postextual

% ---- Refer\^{e}ncias bibliogr\'{a}ficas ----
\bibliography{tese}

% ---- Ap\^{e}ndices ----

% ---
% Inicia os ap\^{e}ndices
% ---
%\begin{apendicesenv}
%
%% Imprime uma p\'{a}gina indicando o in\'{\i}cio dos ap\^{e}ndices
%\partapendices
%
%% ----------------------------------------------------------
%\chapter{Quisque libero justo}
%% ----------------------------------------------------------
%
%\lipsum[50]
%
%% ----------------------------------------------------------
%\chapter{Nullam elementum urna vel imperdiet sodales elit ipsum pharetra ligula
%ac pretium ante justo a nulla curabitur tristique arcu eu metus}
%% ----------------------------------------------------------
%\lipsum[55-57]
%
%\end{apendicesenv}
% ---

% ---- Anexos ----

% ---
% Inicia os anexos - opcional
% ---
\begin{anexosenv}

% Imprime uma p\'{a}gina indicando o in\'{\i}cio dos anexos
\partanexos

\chapter{NetPDL packet description example}
\label{annex:NetPDLdesc}

\begin{lstlisting}[frame=single,language=XML,breaklines=true, tabsize=2,showspaces=false,showstringspaces=false]  % Start your code-block

<protocol name="udp" longname="UDP (User Datagram protocol)"         showsumtemplate="udp">
	<format>
		<fields>
			<field type="fixed" name="sport" longname="{0x8000 15}" size="2" showtemplate="FieldDec"/>
			<field type="fixed" name="dport" longname="{0x8000 16}" size="2" showtemplate="FieldDec"/>
			<field type="fixed" name="len" longname="Payload length" size="2" showtemplate="FieldDec"/>
			<field type="fixed" name="crc" longname="Checksum" size="2" showtemplate="FieldHex"/>
		</fields>
	</format>
	<visualization>
		<showsumtemplate name="udp">
			<section name="next"/>
			<text value="UDP: port "/>
			<protofield name="sport" showdata="showvalue"/>
			<text value=" => "/>
			<protofield name="dport" showdata="showvalue"/>
		</showsumtemplate>
	</visualization>
</protocol>     

\end{lstlisting}

% ---
\chapter{Publications}
% ---
\label{AnnexB}
Three papers were published during this work and are listed below. 

\begin{itemize}

    \item Eder Leão Fernandes, Christian Esteve Rothenberg. "OpenFlow 1.3 Software Switch". In Salão de Ferramentas XXXII Simpósio Brasileiro de Redes de Computadores - SBRC'2014, Florianópolis, 5 a 9 de Maio de 2014.

    \item  E. L. Fernandes, C. Esteve Rothenberg and M. R. Salvador. "Software Defined Home Networking: Research Challenges and Innovation Opportunities."(invited paper), In International Workshop on Telecommunications (IWT'13), Santa Rita do Sapucaí, Brazil, 6-9 May 2013
    
    \item Rodrigo R. Denicol, Eder L. Fernandes, Christian E. Rothenberg, Zoltán Lajos Kis, "On IPv6 support in OpenFlow via Flexible Match Structures". OFELIA/CHANGE Summer School SummerSchool, Poster session, Berlin, Germany, November 7-11 November 2011.
    

\end{itemize}

\chapter{Full Ryu Certification test results }
\label{annex:ryucert}

\begin{minipage}[t][5cm][b]{\textwidth}
\begin{tabular}{|l|l|l|}
\hline
\textbf{Ryu Certification Resume} &  &  \\ \hline
 &  &  \\ \hline
\textbf{ofsoftswitch13} & OK & ERROR \\ \hline
Action & 50 & 6 \\ \hline
(Required) & (3) & (0) \\ \hline
(Optional) & (47) & (6) \\ \hline
set\_field & 159 & 7 \\ \hline
(Optional) & (159) & (7) \\ \hline
Match & 708 & 6 \\ \hline
(Required) & (108) & (0) \\ \hline
(Optional) & (600) & (6) \\ \hline
Group & 15 & 0 \\ \hline
(Required) & (3) & (0) \\ \hline
(Optional) & (12) & (0) \\ \hline
Meter & 30 & 6 \\ \hline
(Optional) & (30) & (6) \\ \hline
Total & 962 & 25 \\ \hline
(Required) & (114) & (0) \\ \hline
(Optional) & (848) & (25) \\ \hline
\end{tabular}
\end{minipage} 

\include{RyuCertification-ofsoftswitch13}

\label{annex:NetPDLdesc}

\end{anexosenv}

% ---- INDICE REMISSIVO ----

\printindex

\end{document} 
